%%%%%%%%%%%%%%%%%%%%%%%%%%%%%%%%%%%%%%%%%%%%%%%%%%%%
% Artifact Appendix Template for Usenix Security'25 AE
%%%%%%%%%%%%%%%%%%%%%%%%%%%%%%%%%%%%%%%%%%%%%%%%%%%%

\appendix
\section{Artifact Appendix}
% \textit{This artifact appendix is meant to be a self-contained document which
% describes a roadmap for the evaluation of your artifact. It should include a
% clear description of the hardware, software, and configuration requirements. In
% case your artifact aims to receive the functional or results reproduced badge,
% it should also include the major claims made by your paper and instructions on
% how to reproduce each claim through your artifact. Linking the claims of your
% paper to the artifact is a necessary step that ultimately allows artifact
% evaluators to reproduce your results.}

% \textit{Please fill all the mandatory sections, keeping their titles and
% organization but removing the current illustrative content, and remove the
% optional sections where those do not apply to your artifact.}

%%%%%%%%%%%%%%%%%%%%%%%%%%%%%%%%%%%%%%%%%%%%%%%%%%%%%%%%%%%%%%%%%%%%%
\subsection{Abstract}
% {\em [Mandatory]}
% {\em Provide a short description of your artifact.}

Our paper presents the first study of China's new capability
to censor QUIC connections based on the Server Name Indication (SNI)
field. We find that China has started decrypting QUIC
Initial packets at scale, employing unique filtering rules
and a distinct blocklist different from its other censorship
mechanisms. We measure the blocked domains, reverse-engineer
the filtering rules (e.g., filtering connections where source port > destination port),
and demonstrate how modest QUIC traffic surges can overwhelm
the system and reduce the effectiveness of its blocking.
Furthermore, we show how the GFW's QUIC blocking mechanism
can be exploited to block UDP connections between arbitrary hosts. 

This artifact provides the code and data from our experiments.
It is structured to make all figures and results in the paper
reproducible and to enable independent verification of our claims
about the GFW's QUIC-SNI blocking.

%%%%%%%%%%%%%%%%%%%%%%%%%%%%%%%%%%%%%%%%%%%%%%%%%%%%%%%%%%%%%%%%%%%%%
\subsection{Description \& Requirements}

% \textit{[Mandatory] This section should list all the information necessary to
% recreate the same experimental setup you have used to run your artifact. Where
% it applies, the minimal hardware and software requirements to run your artifact.
% It is also very good practice to list and describe in this section benchmarks
% where those are part of, or simply have been used to produce results with, your
% artifact.}

\subsubsection{Security, privacy, and ethical concerns}
% \textit{[Mandatory] Describe any risk for evaluators while executing your
% artifact to their machines security, data privacy or others ethical concerns.
% This is particularly important if destructive steps are taken or security
The tools used in this artifact do not pose any risk to the evaluators' machines
or data privacy. All measurements performed using this artifact are non-destructive and
are for verification purposes only.


\subsubsection{How to access}
% {\em [Mandatory]} \textit{Describe here how to access your artifact. If you are
% applying for the Artifacts Available badge, the archived copy of the artifacts
% must be accessible via a stable reference or DOI. For this purpose, we recommend
% Zenodo, but other valid hosting options include institutional and
% third-party digital repositories (e.g., FigShare, Dryad, Software
% Heritage, GitHub, or GitLab — not personal webpages). For repositories that can
% evolve over time (e.g., GitHub), a stable reference to the evaluated version
% (e.g., a URL pointing to a commit hash or tag) rather than the evolving version
% reference (e.g., a URL pointing to a mere repository) is required. Note that the
% stable reference provided at submission time is for the purpose of Artifact
% Evaluation. Since the artifact can potentially evolve during the evaluation to
% address feedback from the reviewers, another (potentially different) stable
% reference will be later collected for the final version of the artifact (to be
% included here for the camera-ready version).}
The repository containing the code and data can be cloned from GitHub:\@
\url{https://github.com/gfw-report/usenixsecurity25-quic-sni}. Archived versions
of the artifact are available under the DOI: 10.5281/zenodo.15626727.

\subsubsection{Hardware dependencies}
% {\em [Mandatory]} \textit{Describe any specific hardware features required to
% evaluate your artifact (vendor, CPU/GPU/FPGA, number of processors/cores,
% microarchitecture, interconnect, memory, hardware counters, etc). If your
% artifact requires special hardware, please provide instructions on how to gain
% access to the hardware. For example, provide private SSH keys to access the
% machines remotely. Please keep in mind that the anonymity of the reviewers needs
% to be maintained and you may not collect or request personally identifying
% information (e.g., email, name, address). [Simply write "None." where this does
% not apply to your artifact.]}
For the AE reviewers,
we have provisioned two Virtual Private Servers (VPSes) for experiments.
Both servers are configured with a single CPU core and 2GB of RAM. 
The first is hosted in
China at the Tencent Cloud Guangzhou Datacenter
(AS45090) and uses an Intel Xeon Platinum 8255C CPU.
The second is in the U.S. at AWS Oregon (AS16509) and
uses an Intel Xeon E5-2686 v4 CPU. Reviewers can SSH into the servers 
using the credentials provided.


\subsubsection{Software dependencies}
% {\em [Mandatory]} \textit{Describe any specific OS and software packages
% required to evaluate your artifact. This is particularly important if you share
% your source code and it must be compiled or if you rely on some proprietary
% software that you cannot include in your package. In such a case, you must
% describe how to obtain and to install all third-party software, data sets, and
% models. [Simply write "None." where this does not apply to your artifact.]}
Both servers provided to the reviewers run Ubuntu 20.04 LTS.
The following software tools and libraries are required
on the local machine to compile and run the
tools and experiments in this artifact:

% Decrease the space between items in the list
\begin{compactitem}
    \item GNU make utility
    \item GNU plot (gnuplot) utility
    \item Docker >= 20.10.0
    \item Python 3.9+
\end{compactitem}

The primary evaluation method uses a Docker container to
build all necessary tools. However, individual tools
can be compiled and run directly on your local machine
without Docker. If you choose this path,
you will also need to install:
\begin{compactitem}
    \item Go 1.22+
    \item Rust (rustc,cargo) 1.82+
    \item Packages: \texttt{build-essential} \texttt{clang} \texttt{cmake}
\end{compactitem}

\subsubsection{Benchmarks}
None.

%%%%%%%%%%%%%%%%%%%%%%%%%%%%%%%%%%%%%%%%%%%%%%%%%%%%%%%%%%%%%%%%%%%%%
\subsection{Set-up}

% {\em [Mandatory]} \textit{This section should include all the installation and
% configuration steps required to prepare the environment to be used for the
% evaluation of your artifact.}
We provide AE reviewers with two pre-configured VPSes (usenix-ae-us and usenix-ae-cn)
that include all the required tools and dependencies.
For those who wish to set up their own environment,
we also offer a one-click script to automatically
compile and transfer the tools on new VPSes.


\subsubsection{Installation}
% {\em [Mandatory]} \textit{Instructions to download and install dependencies as
% well as the main artifact. After these steps the evaluator should be able to run
% a simple functionality test.}

\begin{compactitem}
    \item Docker: \url{https://docs.docker.com/get-docker/}
    \item GNU make: \url{https://gnu.org/software/make/}
    \item Download the repository: \url{https://github.com/gfw-report/usenixsecurity25-quic-sni} using Git clone.
    \item Compile and transfer tools to VPSes:
\begin{verbatim}
cd usenixsecurity25-quic-sni/utils;
make  \
  SERVER_HOST=$USENIX_AE_US \
  SERVER_USER=$USENIX_AE_US_USER \
  SERVER_SSH_KEY=$USENIX_AE_SSH_KEY \
  CLIENT_HOST=$USENIX_AE_CN \
  CLIENT_USER=$USENIX_AE_CN_USER \
  CLIENT_SSH_KEY=$USENIX_AE_SSH_KEY

\end{verbatim}
\end{compactitem}

Running the \texttt{make} command will compile the necessary tools and transfer tools
to the servers. The \texttt{SERVER\_HOST/USER/KEY} and \texttt{CLIENT\_HOST/USER/KEY} variables
should be set using the credentials provided.



\subsubsection{Basic Test}
% {\em [Mandatory]} \textit{Instructions to run a simple functionality test. Does
% not need to run the entire system, but should check that all required software
% components are used and functioning fine. Please include the expected successful
% output and any required input parameters.}

Log in to the two VPSes using the provided credentials:
\begin{verbatim}
ssh -i $USENIX_AE_SSH_KEY \ 
  $USENIX_AE_CN_USER@$USENIX_AE_CN

# In a separate terminal window:
ssh -i $USENIX_AE_SSH_KEY \
  $USENIX_AE_US_USER@$USENIX_AE_US
\end{verbatim}


\noindent Then run the following command on the usenix-ae-us VPS:
\begin{verbatim}
sudo tcpdump udp and host $USENIX_AE_CN \
 -Uw - | ./server-parser
\end{verbatim}     


\noindent On the usenix-ae-cn VPS:
\begin{verbatim}
echo baidu.com | ./quic-sni-sender  -p 443 \ 
  --sport=65000 --dip=$USENIX_AE_US \ 
  --socket-pool-size 1
\end{verbatim}

\noindent After 10 seconds, check the terminal on the server (usenix-ae-us).
You will see output in CSV format. The value in the last
column indicates the test result. A value of \texttt{True} means the
QUIC connection was blocked, while \texttt{False} means
the connection was successful.




%%%%%%%%%%%%%%%%%%%%%%%%%%%%%%%%%%%%%%%%%%%%%%%%%%%%%%%%%%%%%%%%%%%%%
\subsection{Evaluation workflow}
% {\em [Mandatory for Artifacts Functional \& Results Reproduced, optional for
% Artifact Available]} \textit{This section should include all the operational
% steps and experiments which must be performed to evaluate if your your artifact is
% functional and to validate your paper's key results and claims. For that
% purpose, we ask you to use the two following subsections and cross-reference the
% items therein as explained next.}

\subsubsection{Major Claims}
% {\em [Mandatory for Artifacts Functional \& Results Reproduced, optional for
% Artifact Available]} \textit{Enumerate here the major claims (Cx) made in your
% paper. Follows an example:}\\

\begin{compactdesc}
    \item[(C1):] GFW decrypts QUIC Initial packets to inspect SNI fields and block QUIC connections.
    This is proven by experiment (E1).
    \item[(C2):] The GFW blocks QUIC connections where the client's
    source port is greater than the server's destination port.
    This is supported by experiment (E2).
    \item[(C3):] The GFW exhibits varied responses to different QUIC and QUIC-like payloads. This can be 
    confirmed by experiment (E3). 
    \item[(C4):] Sending a dummy UDP payload before the Client
    Initial packet can bypass the GFW's QUIC-SNI blocking. This is proven by experiment (E4).

\end{compactdesc}

\subsubsection{Experiments}

Before proceeding, ensure the VPSes
are set up with the tools and dependencies detailed in
the previous section. The experiments will be conducted between a
client machine in China (usenix-ae-cn) and a server in the U.S. (usenix-ae-us).
In each experiment, the client will send QUIC packets/probes to the server.
To prevent residual interference, a 5-minute interval should be observed between
each test. We recommend running the experiments multiple times, as
the GFW's QUIC-blocking behavior can exhibit diurnal variations (see Section 3.4 of the paper).

% use paralist for more compact list format: for more details check here:
% https://texfaq.org/FAQ-complist
\begin{compactdesc}

    \item[(E1):] \textit{[Test QUIC-SNI Blocking] [5 human-minutes + 5 compute-mintues]:
    This experiment verifies the GFW's ability to decrypt and block QUIC connections based on SNI in the Initial Packet.}

    \begin{asparadesc}
        % \item[How to:]  \textit{Describe thoroughly the steps to perform the
        % experiment and to collect and organize the results as expected from your
        % paper. We encourage you to use the following structure with three main
        % blocks for the description of your experiment.}

        \item[Preparation:] Log in to the two VPSes using the provided credentials.
        

\item[Execution:] On the usenix-ae-us VPS, run:
\begin{verbatim}
sudo tcpdump udp and host $USENIX_AE_CN \
-Uw - | ./server-parser
\end{verbatim}
On the client (usenix-ae-cn), run the following command to send a QUIC probe. 
This probe consists of a Client Initial packet, followed by a 5 second delay, and then subsequent 10-byte UDP payloads.
\begin{verbatim}
echo google.com | ./quic-sni-sender -p 443 \
--socket-pool-size 1 --dip=$USENIX_AE_US \
--sport=55000
\end{verbatim}
\vspace{3pt}
\item[Results:] Wait approximately 10 seconds for the server-parser script on the server to produce its output. A typical result will appear as follows:
\vspace{3pt}
\begin{tiny}
\begin{verbatim}
# 2025/06/04 03:01:34 Started parsing: /dev/stdin
# tcpdump: listening on eth0, link-type EN10MB (Ethernet), snapshot length 262144 bytes
# 2025-06-04T03:02:08Z,{usenix-ae-cn_IP},{usenix-ae-us_IP},55001,4437,google.com,true
\end{verbatim}
\end{tiny}
The parser determines that a connection is blocked if it receives the QUIC Initial Packet but none
of the subsequent UDP payloads arrive. This outcome is indicated by the true value in the final column. 
A false value signifies that no blocking was detected.
Test other SNI values like \texttt{baidu.com} (to verify behavior for exempt domains) and
\texttt{cloudflare-dns.com}, \texttt{youtube.com} (to verify blocking for other domains).

\end{asparadesc}



\item[(E2):] \textit{[Testing Rule: SourcePort > Destination Port] [5 human-minutes + 5 compute-minutes]: This experiment tests the GFW's filtering rule for QUIC Initial packets in which the source port must be greater than the destination port.}
\begin{asparadesc}
    \item[Preparation:] Run the parser \texttt{( sudo tcpdump udp and host \$USENIX\_AE\_CN -Uw - | ./server-parser )} from E1 on the server (usenix-ae-us). 
    \item[Execution:] Run the following command on the client (usenix-ae-cn):

\begin{verbatim}

echo google.com | ./quic-sni-sender \ 
-p 5000 --socket-pool-size 1 \ 
--dip=$USENIX_AE_US \
--followup-payloads=10 \ 
--bind-ip=0.0.0.0 \
--sport=4000 \
--no-use-greater-srcports 
# SrcPort (4000) < DestPort (5000) 
-> Expected Result: NOT BLOCKED


echo google.com | ./quic-sni-sender \ 
-p 443 --socket-pool-size 1 \ 
--dip=$USENIX_AE_US \
--followup-payloads=10 \ 
--bind-ip=0.0.0.0 \
--sport=65000 
# SrcPort (65000) > DestPort (443) \ 
-> Expected Result: BLOCKED

\end{verbatim}
        \item [Results] The results from the first command should show a \texttt{False} value in the last column, indicating that the connection was not blocked. The second command should show a \texttt{True} value, indicating that the connection was blocked due to the source port being greater than the destination port. 


    \end{asparadesc}

    \item[(E3):] \textit{[Test Different QUIC Payloads] [30 human-minutes + 30 compute-minutes]: This experiment evaluates the GFW's blocking behavior based on various QUIC payloads as detailed in Table 3 of the paper. The payloads can be generated using the \texttt{quic-packet-builder} utility in the \texttt{utils} directory.}
    
    \begin{asparadesc}
        \item[Preparation:] Begin by capturing UDP traffic on the server (usenix-ae-us) using: \texttt{sudo tcpdump udp and host \$USENIX\_AE\_CN} 
        \item[Execution:] On the client (usenix-ae-cn), run the following commands using the payloads files linked in the table \href{https://github.com/gfw-report/usenixsecurity25-quic-sni?tab=readme-ov-file#experiment-3-testing-quic-payloads-listed-in-table-3-of-the-paper}{here}.
        
\begin{verbatim}
# Send QUIC payload using netcat
nc -u -q 0 -p 60001 $USENIX_AE_US 444 \ 
< ./payloads/exp1.bin
\end{verbatim}
Then send follow-up payloads repeatedly to the same destination port using the same source port using the following command:
\begin{verbatim}
# Send arbitrary UDP payload to the server
echo "10101010101010101010" | xxd -r -p \
| nc -u -q 0 -p 60001 $USENIX_AE_US 444
\end{verbatim}

        \item[Results:] Depending on if any follow-up payload packets arrive at the server, one can determine if the connection was blocked or not. The results should match findings in Table 3 of the paper. Change the source and destination ports between each test to avoid residual interference.
    \end{asparadesc}





    \item[(E4):] \textit{[Test Dummy Payload Bypass] [5 human-minutes + 5 compute-minutes]: This experiment shows how sending a dummy UDP payload before the Client Initial packet can bypass the GFW's blocking.}
    \begin{asparadesc}
        \item[Preparation:] Capture UDP traffic on the server (usenix-ae-us) using: \texttt{sudo tcpdump udp and host \$USENIX\_AE\_CN }
        \item[Execution:] Run the following command on the client (usenix-ae-cn):
\begin{verbatim}
# Send an arbitrary UDP payload to the server
echo "0000000000000000000000000000" \
| xxd -r -p | nc -u -q 0 \
-p 65535 $USENIX_AE_US 6126
\end{verbatim}
Wait a few seconds and then send the QUIC packet containing a forbidden SNI (google.com in this example):
\begin{verbatim}
echo google.com | ./quic-sni-sender -p 6126 \
--socket-pool-size 1 --dip=$USENIX_AE_US \
--sport=65535
\end{verbatim}
        \item[Results:] Successful arrival of all subsequent packets on the 4-tuple \texttt{(src\_ip, 65535, dst\_ip, 6126)} at the server confirms the bypass. 
    \end{asparadesc}

\end{compactdesc}



%%%%%%%%%%%%%%%%%%%%%%%%%%%%%%%%%%%%%%%%%%%%%%%%%%%%%%%%%%%%%%%%%%%%%

\begin{table*}[t]
    \centering
    \small
    \begin{tabularx}{\linewidth}{|X|X|X|X|} 
        \hline
    \textbf{Section} & \textbf{Figure} & \textbf{Command} & \textbf{Output file} \\

    \hline
    3.2 GFW's Blocking Latency & Fig 2 \& 10 & cd how-fast-gfw-blocks; make clean; make & how-fast-the-gfw-blocks.pdf, how-fast-the-gfw-blocks-boxplot.pdf \\
    \hline
    3.3 Blocking Rule: Source Port \textgreater{}= Destination Port & Fig 3 \& 11 & cd rule-srcport-greater-than-dst-port; make clean; make & heatmap-ports-401-450-step-1\_heatmap.pdf, heatmap-ports-1-65000-step-1000\_heatmap.pdf \\
    \hline
    3.4 Diurnal Blocking Pattern & Fig 4 & cd diurnal-blocking; make clean; make & diurnal-timeseries-three-sources.pdf \\
    \hline
    3.6 Parsing Idiosyncrasies & Fig 5 & cd what-triggers-blocking; make clean; make & quic\_parse\_heatmap.pdf \\
    \hline
    4. Monitoring the Blocklist over Time & Fig 6 & cd sni-blocklist; make clean; make & domains-blocked-over-quic-weekly.pdf \\
    \hline
    4.1. Comparison with Other Blocklists & Fig 7 & cd overlap-between-blocklists; make clean; make & venn-intersection-between-lists.pdf \\
    \hline
    5. GFW Degradation Attack & Fig 8 & \small{cd degradation-attack; cd Figure\_8\_experiments /AVG\_exp23-22-20\_sensitive-stressing\_and\_exp25-26-27\_random-stressing; make clean; make} & stressing\_rates.eps \\
    \hline
    \end{tabularx}
    \label{tab:t1}

    \caption{This table lists the commands required to reproduce each figure from the main paper using the provided artifact. For each figure, the corresponding section, command, and output file are specified.}
\end{table*}




\begin{table*}[h!]
    \centering
    \small
    \begin{tabularx}{\linewidth}{|l|X|X|} 
    \hline
    \textbf{Table} & \textbf{Data sources} & \textbf{Command} \\
    \hline
    \textbf{1} & ./network-tap/data/tuple-count-2025-01-22-16-00.statistics-quic-conn.txt, ./network-tap/data/tuple-count-2025-01-22-16-00.statistics-udp-pkt.txt & — \\
    \hline
    \textbf{2} & ./ttl-location/data/DNS/\{city\}-dns-and-traceroute-result.txt, ./ttl-location/data/QUIC/\{city\}-ttl\_anon.pcap & — \\
    \hline
    \textbf{3} & ./what-triggers-blocking/payloads, ./what-triggers-blocking/results.txt & — \\
    \hline
    \textbf{4} & ./sni-blocklist/ & \verb|make clean && make| \\
    \hline
    \textbf{5} & ./overlap-between-blocklists & \verb|make clean && make| \\
    \hline
    \textbf{6} & ./availability-attack/data/ec2/ & \verb|make clean && make| \\

    \hline
    \end{tabularx}
    \label{tab:t2}

    \caption{This table lists the data sources and commands required to reproduce each table from the main paper using the artifact repository.}
\end{table*}


\subsection{Reproducing Paper Resources}
\label{sec:reuse}
% {\em [Optional]} \textit{This section is meant to optionally share additional
% information on how to use your artifact beyond the research presented in your
% paper. In fact, a broader objective of an artifact evaluation is to help you
% make your research reusable by others.}

% \textit{You can include in this section any sort of instruction that you believe
% would help others re-use your artifact, like, for example, scaling down/up
% certain components of your artifact, working on different kinds of input or
% data-set, customizing the behavior replacing a specific module/algorithm, etc.}

To reproduce the figures and data presented in the paper, follow these instructions:
First, set up your Python environment:
\begin{verbatim}
python3 -m venv venv
. venv/bin/activate
pip install -r requirements.txt
\end{verbatim}


Each figure and table in the paper can be reproduced
individually via the make command in its respective experiment directory.
A comprehensive list of the commands required to reproduce each figure and table is provided in
Tables 1 and 2. These commands should be run from the \texttt{experiments} directory.








%%%%%%%%%%%%%%%%%%%%%%%%%%%%%%%%%%%%%%%%%%%%%%%%%%%%%%%%%%%%%%%%%%%%%

\subsection{Version}
%%%%%%%%%%%%%%%%%%%%
% Obligatory.
% Do not change/remove.
%%%%%%%%%%%%%%%%%%%%
Based on the LaTeX template for Artifact Evaluation V20231005. Submission,
reviewing and badging methodology followed for the evaluation of this artifact
can be found at \url{https://secartifacts.github.io/usenixsec2025/}.